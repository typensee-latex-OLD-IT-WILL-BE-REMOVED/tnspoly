\documentclass[12pt,a4paper]{article}

\makeatletter
	\input{../config/header[fr].sty}

	\usepackage{01-polynomial-and-co}
\makeatother



\begin{document}

\section{Polynômes}

\newparaexample{Polynômes}

\begin{latexex}
$\setpoly{R}{X}$ ou
$\setpoly{R}{X | Y | Z}$
\end{latexex}


% ---------------------- %


\newparaexample{Fractions polynômiales}

\begin{latexex}
$\setpolyfrac{Q}{T}$ ou
$\setpolyfrac{Q}%
             {S_1 | S_2 | \dots | S_k}$
\end{latexex}




% ---------------------- %


\subsection{Fiches techniques}

\paragraph{Polynômes}

\IDmacro*{setpoly}{2}

\IDmacro*{setpolyfrac}{2}


% ---------------------- %


\section{Séries formelles classiques}

\newparaexample{Séries formelles}

\begin{latexex}
$\setserie{C}{X}$ ou
$\setserie{C}{T | O | P}$
\end{latexex}


% ---------------------- %


\newparaexample{Corps des fractions de séries formelles}

\begin{latexex}
$\setseriefrac{Z}{X}$ ou
$\setseriefrac{Z}{Z | T | O | P}$
\end{latexex}


% ---------------------- %


\subsection{Fiches techniques}

\paragraph{Séries formelles classiques}

\IDmacro*{setserie}{2}

\IDmacro*{setseriefrac}{2}


% ---------------------- %


\section{Polynômes de Laurent et séries formelles de Laurent}

\newparaexample{Polynômes de Laurent}

Ci-dessous, la notation $\setpolylaurent{R}{X_1 | X_2}$ n'est pas standard.

\begin{latexex}
$\setpolylaurent{R}{X} =
 \setpoly{R}{X | X^{-1}}$

$\setpolylaurent{R}{X_1 | X_2} =
 \setpoly{R}{X_1 | X_1^{-1} %
           | X_2 | X_2^{-1}}$
\end{latexex}


% ---------------------- %


\newparaexample{Séries formelles de Laurent}

Ci-dessous, la notation $\setserielaurent{Q}{X_1 | X_2}$ n'est pas standard.

\begin{latexex}
$\setserielaurent{Q}{X} =
 \setserie{Q}{X | X^{-1}}$

$\setserielaurent{Q}{X_1 | X_2} =
 \setserie{Q}{X_1 | X_1^{-1} %
            | X_2 | X_2^{-1}}$
\end{latexex}


% ---------------------- %


\subsection{Fiches techniques}

\paragraph{Polynômes de Laurent et séries formelles de Laurent}

\IDmacro*{setpolylaurent}{2}

\IDmacro*{setserielaurent}{2}

\IDarg{1} l'ensemble auquel les coefficients appartiennent.

\IDarg{2} cet argument est une suite de "morceaux" séparés par des barres \verb+|+, chaque morceau étant une variable formelle.

\end{document}
